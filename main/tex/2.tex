%!TEX root = ../main.tex

\section{Threshold PSI with associated data} % (fold)
\label{sec:threshold_psi_with_associated_data}

We provide a brief overview of the authors' warmup problem denoted \textbf{threshold PSI with associated data}, or \textbf{tPSI-AD}. We refer to Table~\ref{tbl:notation} for all the relevant notation we use.

Suppose the server has a set $X \of \U$ of size $n$, and the client has an unordered list of $m$ triples:
\begin{align*}
      \bar{Y} = \lp{\lp{y_1, id_1, ad_1} , \dots   \lp{y_m, id_m, ad_m}} \in \lp{\U \times \ID \times \D}^m,
\end{align*}
where $y \in \U$ is the hash of an image, a unique identifier $id \in \ID$, and some associated data $ad \in \D$. Identifiers are not assumed to be secret, and are treated as freshly sampled independent random bit strings. We introduce a threshold parameter $t$ known to both the client and server For tPSI-AD, we aim to design a protocol such that at termination, \begin{enumerate*}[label=(\roman*)]
	\item the client only learns $\norm{X}$, 
	\item the server only learns $\Yb_{id}$, and
\end{enumerate*}
\begin{itemize}
	\item if $\norm{id \lp{\Yb \cap X}} < t$ then the server learns $id \lp{\Yb \cap X} \of \ID$, or
	\item if $\norm{id \lp{\Yb \cap X}} \geq t$ then the server learns $\Yb\lb{id \lp{\Yb \cap X}}_{\lbr{id,ad}} \of \lp{\ID \times \D} $.
\end{itemize}
Namely, if the threshold is not met, the server learns the identifiers in $id \lp{\Yb \cap X}$, but no associated data. Otherwise, the server learns $\Yb\lb{id \lp{\Yb \cap X}}_{\lbr{id,ad}}$, which is the associated data for all the identifiers in the intersection.

\begin{table}[t]
\centering
	\begin{tabular}{c l}
\toprule
	

		\textbf{Symbol} & \textbf{Meaning} \\
		\hline
		\hline
		$\U$	& 	Universe of all possible image hash values\\
		$X \of \U$	&Set of distinct hash values the server has, s.t. $\norm{X} = n$. \\
		$\bar{Y} = \lp{\lp{y_i, id_i, ad_i }}$	& Triples the client has, s.t. $\norm{\bar{Y}} = m, i \in \lb{1,m}$.	\\
		$y \in \mathcal{U}$	& 	Hash value\\
		$id \in \mathcal{ID}$	& 	Unique identifier of a triple\\
		$ad \in \mathcal{D}$	& 	Associated data of a triple\\
		$id(\bar{Y})$& Set of $id$'s of triples in $\bar{Y}$     \\

		$id(\bar{Y} \cap X)$& Set of $id$'s of triples in $\bar{Y}$ whose $y$ is also in $X$     \\
		$\bar{Y}_{id} \in \ID^{m}$	& List of all $id$'s in the triples in $\Yb$    \\
		$\Yb_{id, ad} \of \lp{\ID \times \mathcal{D}} $ & Set of $id$'s and $ad$'s in the triples in $\Yb$  \\
		{$\Yb \lb{T}\of \lp{\U \times \mathcal{ID} \times \mathcal{D}}^{\leq m} $ }	& The list of triples in $\Yb$ whose $id$'s are in $T \of \ID$  \\
$\Yb\lb{id \lp{\Yb \cap X}}_{\lbr{id,ad}} \of \lp{\ID \times \D} $ & Set of associated data for all $id$'s in intersection\\
		$x = d$	&   Assignment of value $d$ to variable $x$  \\
		$x \gets A (\cdot)$	&   $x$ is the output of a randomized algorithm $A$\\
\bottomrule
	\end{tabular}
	\caption{PSI notations.}
	\label{tbl:notation}
\end{table} 

Naturally, the protocol should satisfy the following security goals:
  \begin{itemize}
    \item The server cannot recover the user's matched photos without exceeding the threshold $t$.
    \item False positives are impossible.
    \item The server learns no information is learned about the client's non-matched images.
    \item The user cannot learn any information about $X$ aside from its size.
    \item The user cannot identify which of their images were flagged as CSAM by the system.
\end{itemize}
We also assume both the client and server honestly adhere to the protocol as described. We refer the reader to \textsection4.4 of \cite{bhowmick2021apple} for a complete, concrete security description, including proofs of robustness against malicious clients and servers.


An additional security goal of masking the number of matches a client has accumulated until the threshold is exceeded can be achieved through the fuzzy version of tPSI-AD (aptly named ftPSI-AD). This requires construction of the novel \emph{detectable hash function} primitive, which is beyond the scope of this project.