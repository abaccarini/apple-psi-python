%!TEX root = ../main.tex


\section{Building Blocks} % (fold)
\label{sec:building_blocks}

We define the following cryptographic primitives and their respective constructions below:
\begin{itemize}
	\item $\lp{\enc, \dec}$ is a symmetric encryption scheme with  key space $\K'$ and provides IND-CPA security (see \cite{katz2014introduction} \textsection3.4.2 for the full definition) and \emph{random key robustness}, which states that if $k \neq k'$ are independent random keys, then $\dec(\enc(k,m), k')$ should fail with high probability. AES128-GCM satisfies both requirements.



	\item $\Gdh$ is a Diffie-Hellman group of prime order $q$ with $G$ as a fixed generator and Decision Diffie-Hellman (DDH) assumption holds. We use Group 14 with a 2048-bit modulus, and~$G = 2$.

	\item $H : \U \to \Gdh$ is a hash function modeled as as random oracle. This is implemented using HMAC with SHA256, and converting the output digest to an integer $\pmod q$.
	\item $h : \U \to \lbr{1,\dots,\eta}$ is a random hash. This is implemented using  SHA256 and converting the output digest to an integer $\pmod \eta$.

	\item $H' : \Gdh \to \K'$ is secure key derivation function; the uniform distribution on $\Gdh$ mapped to an ``almost'' uniform distribution on $\K'$. This is implemented using HKDF with SHA256 to produce a 128-bit key.
	\item Shamir secret sharing on an element of $\K'$ to obtain shares in $\F_{\text{Sh}}$ for some field $\F_{\text{Sh}}$ over a prime $\sh$; $\F_{\text{Sh}}$ must be  sufficiently large such that when choosing $t+1$ random elements from $\F_{\text{Sh}}$, the probability of a collision is low.

	\item A pseudorandom function (PRF) $F : \K'' \times \mathcal{ID} \to \F_{\text{Sh}}$. This is also  constructed using HMAC with SHA256, and converting the output digest to an integer $\pmod \sh$.
\end{itemize}



\noindent \textbf{Diffie-Hellman Random Self Reducability}: We introduce a particularly useful property for the protocol as follows.
\begin{definition}
\label{def:dh_tup}
 Let $ \G $ be a group of prime order $q$ with a fixed generator $G \in \G$, and suppose $\lp{L, U, V} \in \G^3$. Then triple $\lp{L,U,V}$ is a \textbf{Diffie-Hellman (DH) tuple} if there exists an $\alpha \in \F_q$ such that $L = G^\alpha$ and $V = U^\alpha$.
 \end{definition} 
\noindent
We work through the arithmetic of a partial random self reduction for DH tuples below. 
\begin{definition}[DH self-reduction]
Given a triple $\lp{L, T, P} \in \G^3$, we
\begin{itemize}
	\item choose a random $\beta, \gamma \in \F_q$,
	\item compute $Q = T^\beta \cdot G^\gamma$ and $S = P^\beta \cdot L^\gamma$,
	\item output $\lp{L,Q,S}$.
\end{itemize}
\end{definition}
\noindent
The transformation $\lp{L, T, P} \to \lp{L, Q, S}$ has the following properties:
\begin{itemize}
	\item If $\lp{L, T, P}$ is a DH tuple where $L = G^\alpha$, then $Q$ is a fresh uniformly sampled element in $\G$, and 
	\begin{align*}
	S =  P^\beta \cdot L^\gamma = \lp{T^\alpha}^\beta \cdot \lp{G^\alpha}^\gamma =  \lp{T^\beta \cdot G^\gamma}^\alpha = Q^\alpha.
	\end{align*}
	\item If $\lp{L, T, P}$ is not a DH tuple, then  $(Q,S)$ is a fresh uniformly sampled pair in $\G^2$.
\end{itemize}
